\documentclass{scrartcl} % https://texdoc.org/serve/scrartcl/0
\KOMAoptions{%
    fontsize=10pt,%
    titlepage=off,%
    headings=big,%
    headsepline=off,%
    paper=a4,%   
    twoside=off,%
    % parskip=half+,%
    % bibliography=totoc,%    
}%

\usepackage[T1]{fontenc}
\usepackage[utf8]{inputenc}
\usepackage[brazil]{babel}
%\usepackage[math]{anttor} % fonte um pouco mais estilizada
\usepackage{import}
%\usepackage{parskip}
\usepackage{fontawesome}
%=========================Packages==================================%
\usepackage{luacode}
\usepackage{lscape,booktabs,latexsym,multicol,lmodern, natbib,graphicx,tikz,tkz-euclide, lipsum,siunitx, setspace,float}
\usepackage{xcolor}
\usepackage{amsmath,amsfonts,amssymb,amsthm}
\everymath{\displaystyle}
\usepackage{enumitem} % númeração dos items
\setlist[enumerate, 1]{label =\textbf{\arabic*.}} % númeração global de itens e subitems, em 1,1.1,1.2,..
\setlist[enumerate, 2]{label =\textbf{\theenumi \arabic*}} % númeração global de itens e subitems, em 1,1.1,1.2,..

% configurações das questões, bem como: pontuação e estrutura.

\usepackage{tasks} % cria lista curta
\usepackage{exsheets} % cria questoes
\SetupExSheets[points]{name=ponto/s,number-format=\color{blue}} % define as configurações de pontuação das questões, e a cor da pontuação.

%\SetupExSheets{headings=fancy-wp} % estilo diferente para o topo do enunciado com o nome " Exercício

\newcommand{\titulo}[1]{%
\begin{center}
    {\LARGE {\scshape #1}}
\end{center}
}

\graphicspath{{imgs/}} %informa a pasta em que as imagens estão
%\usepackage{showframe} %Mostra linhas de marcação e margens
%FONTES
%\usepackage{fontspec}
%\usepackage{avant}
%\usepackage{mathptmx}
\usepackage{times}
%\usepackage[classicRfeIm]{kpfonts}
%\usepackage{kurier}

\usepackage{eso-pic,graphicx} % Para funcionar o background

\usepackage{hyperref}% add hypertext capabilities
%\usepackage{txfonts}
%\usepackage{mathrsfs}

% CORES
\usepackage{xcolor}
\definecolor{corprimaria}{RGB}{10,48,123} 
\definecolor{corsecundaria}{RGB}{116,23,255}
\definecolor{corlinha}{RGB}{0,102,204}

\definecolor{corexercicio}{RGB}{38,38,38}
\definecolor{cordefinicao}{RGB}{47,158,65}
% \definecolor{corexemplo}{RGB}{116,23,255}
\definecolor{corexemplo}{RGB}{38,38,38}

\definecolor{color1}{HTML}{2B323F}
\definecolor{color2}{HTML}{CAEC7D}

%===================================================
% MARGINS
%\usepackage[top=8mm, bottom=20mm, left=8mm, right=8mm]{geometry}
\usepackage{geometry}
\geometry{
	paper=a4paper, 
	top=3cm, 
	bottom=2.5cm, 
	left=1cm, 
	right=1cm,
	headheight=14pt, % Header height
	footskip=1.4cm, % Espaço da margem inferior à linha de base do rodapé
	headsep=10pt, % Espaço da margem superior até a linha de base do cabeçalho
	%showframe, % Uncomment to show how the type block is set on the page
}


% NOVOS COMANDOS
\newcommand{\atv}{Lista de Exercícios 00}
\newcommand{\preceptor}{Monitor: Matheus Jonatha}

%%------------------------------------------
%% CABEÇALHO E RODAPÉ
%%------------------------------------------
\usepackage{fancyhdr}
\pagestyle{fancy}
\renewcommand{\headrulewidth}{0pt} %linha horizontal no topo da pagina
%\renewcommand{\footrulewidth}{0.4pt} %linha horizontal no pé da pagina

% \fancyhead[L]{}
\fancyhead[R]{
% \textbf{Monitoria AnnWay - Curso de Matemática Básica}
    % \begin{tikzpicture}[overlay]
    %     \node[draw=corlinha,
    %     circle,minimum width=.7cm, minimum height=.7cm,
    %     anchor=south west,
    %     fill=corlinha,font=\fontsize{11}{15}\sffamily,inner sep=1pt,outer sep=1pt]
    %     at (0,0){\textcolor{white}{\thepage}};
    % \end{tikzpicture}
}
% \fancyhead[C]{}

\fancyfoot[R]{
    \begin{tikzpicture}[overlay]
        \node[draw=corlinha,
        circle,minimum width=.7cm, minimum height=.7cm,
        anchor=south west,
        fill=corlinha,font=\fontsize{11}{15}\sffamily,inner sep=1pt,outer sep=1pt]
        at (0,0){\textcolor{white}{\bfseries \thepage}};
    \end{tikzpicture}
}
\fancyfoot[C]{}

%\setlength\parindent{0pt}
%\setlength\parskip{1.5ex}
%\setlength\parsep{1.5\parskip}
%\thispagestyle{empty}%\bigskip %Rodapé na primeira pagina


%para nao ficar o retangulo em volta dos links, apenas muda a cor dos caracteres
\hypersetup{colorlinks,
linkcolor=blue,
filecolor=blue,
urlcolor=blue,
citecolor=blue }



\hypersetup{pdfauthor={Matheus Jonatha (@mathjonatha)},
            pdftitle={Lista de Exercícios – Aula 01 - Matemática Aplicada},
            pdfsubject={Matemática Aplicada - Janiheryson Felipe De Oliveira Martins},
            pdfkeywords={mathjonatha,mthsjonatha},
            pdfproducer={Produzido e compilado no overleaf.com},
            pdfcreator={LuaLaTeX}}
            
       

\SetupExSheets{solution/print=true} % Mostra a solução das questões quando print = true

% \usepackage[use-files]{xsim}



\begin{document}


\begin{center}
\textbf{Lista de Exercícios - Aula 1}
\end{center}

\begin{multicols}{2}

\setlength\columnseprule{1pt}
\def\columnseprulecolor{\color{corlinha}}%

\begin{question}
Como o salgadinho sofreu um aumento após a redução da quantidade das gramas no pacote, o preço continua o mesmo valor de R\$ 3,00.

Considerando qual seria o seu novo valor sem o aumento, teríamos que:

\begin{align*}
90x &= 80 \cdot 3   \\
x &= \frac{240}{90} \\
x &\approx 2,7
 \end{align*}
 
Agora fazendo o aumento de 12,5\%, teremos que:
\begin{align*}
   100x &= 2,7 \cdot 112,5\\
   x &= \frac{300}{100} \\
   x &= 3
\end{align*}
Logo, temos que o valor se mantém o mesmo após as alterações, continuando R\$ 3,00.
\end{question}


\begin{question}
Para esse caso temos que achar o tempo de maneira individual, tanto para os 30\% quanto para o 70\%. Ao fim somando seus tempos. Teremos então que:

\begin{align*}
\frac{1}{8}&= \frac{x}{20} \\
x&= \frac{20}{8} \\
x &= 2,5\\
 \end{align*}
Assim encontramos já que o tempo para os oito núcleos executar essa tarefa é de 2,5 segundos. Para finalizar, é preciso encontrar o tempo equivalente utilizado nos porcentagens informadas no enunciados. Assim temos:

\begin{align*}
\textrm{Tempo total de processamento} &= 2,5 \cdot \frac{30}{100} + 20 \cdot \frac{70}{100}\\
\textrm{Tempo total de processamento} &= 14,75 \textrm{ segundos}\\
 \end{align*}
 
Por fim, concluímos que o tempo total foi de 14,75 segundos para executar essa tarefa utilizando os oito núcleos.
\end{question}

\begin{question}
Como temos que a constante de proporcionalidade é \(\frac{x}{y} = k\). Podemos encontrar fazendo a relação final informada no enunciado, onde teremos que fazer entre o tempo gasto na ordenação(\(t\)) e o número de elementos(\(n\)) da lista.\par
Inicialmente temos que \(k = 0,00015\), pois  é \(\frac{t}{n}\). Ainda podemos escrever como sendo \(k = \frac{15}{10^5}\), disso temos que a função de proporcionalidade pode ser:

\[f(n) = n \cdot \frac{15}{10^5}\]

Ou ainda,

\[f(n) = n \cdot 0,00015\]

\end{question}
% \break
% \columnbreak
\end{multicols}

\newpage

\begin{center}
\textbf{Lista de Exercícios - Aula 2}
\end{center}


% \begin{center}
% \textbf{Lista de Exercícios - Aula 2 }
% \end{center}

\begin{multicols}{2}
\setlength\columnseprule{1pt}
\def\columnseprulecolor{\color{corlinha}}%

\begin{question}
Para os dados informados temos que o único valor que é inversamente proporcional será o número de torneiras. Sendo assim podemos fazer já de maneira direta que:

\begin{align*}
\frac{18}{x} &= \frac{1000 \cdot 5}{5000 \cdot 2}\\
\frac{18}{x}&= \frac{5000}{8000}\\
18 \cdot 8000&= 5000 \cdot x \\
x &=  \frac{144 000}{5000}\\
x&= 28,8\\
 \end{align*}
 
Para fim, convertendo para minutos e segundos, temos que 28,8 será exatamente \textbf{28 minutos e 48 segundos} para que o tanque com 4000 L seja completamente cheio com cinco torneiras abertas.
\end{question}

\begin{question}
Usando a mesma ideia da questão anterior, temos que nesse caso todos os valores são proporcionais. Com isso temos que:

\begin{align*}
\frac{4800}{x}&= \frac{20 \cdot 30}{ 65 \cdot 100} \\
x \cdot 600 &= 6500 \cdot 4800\\
x &=  \frac{6500 \cdot 4800}{600}\\
x &= \frac{650 \cdot 480}{6}\\
x & = 52 000
 \end{align*}
 Temos assim que a empresa gastará \textbf{52 mil reais} com alimentação para os 65 funcionários nesses 100 dias.
\end{question}

\begin{question}
Nesse caso temos que a grandeza inversamente proporcional nesse caso será o número de pessoas, assim resolvendo como das maneiras anteriores, temos que:

\begin{align*}
\frac{15}{x} &= \frac{30 \cdot 100}{ 20 \cdot 300} \\
\frac{15}{x}&= \frac{1}{2} \\
x &=  30\\
 \end{align*}

Por fim, temos que será necessário \textbf{30 dias} para as 30 pessoas consumirem os 300kg de arroz.
\end{question}

\end{multicols}


\end{document}
